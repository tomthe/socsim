
\subsection{Event competition}
\label{sec:eventcomp}
At the start of every month in the simulation, every living person has
exactly one event scheduled for some future date. In the course of
this month, all scheduled events are executed.  Events affect the
individuals for whom they are scheduled but may also affect spouses
and others to whom they are connected. All births, deaths, marriages and
group transitions that are scheduled for the current month are
executed in random order.  After a person's event is executed, unless
that event was death, a new event must be scheduled. 
New events are scheduled 
by an ``event competition.'' This event competition is also held once
for each living person at the beginning of each simulation segment
(that is, every time the demographic rates or societal constants
change).

Each event for which the individual is at risk (e.g., men rarely give
birth) can be modeled as a piecewise exponential distribution. A
random number is used to generate a waiting time until this event
occurs (which is bounded by the individual's maximum possible age at
death). The individual's next event is the one with the shortest
randomly generated waiting time. The event competition thus follows a
competing risk framework wherein the probability of each event is
independent of all others.


\subsection{Generating potential waiting times}
\label{sec:datev}

The waiting time algorithm is conceptually equivalent to drawing a
random number u, from a uniform (0,1) distribution, calling u the
probability that the event will not yet have occurred, then finding
the first month by which the probability of non-occurrence is less
than or equal to u. The probability that an event will not have
occurred by a particular month T is given by the expression

\begin{equation}
  \label{eq:probnon}
  \prod_{t=0..T} (1-p_{t})
\end{equation}



Where $p_{t}$ is the probability of the event's occurrence in period t
conditioned on it not having occurred at any time before t. Since
$(1-p_{t})$ is always between 0 and 1, the expression given above is
nonincreasing in $T$. Consequently, beginning with $t = 0$ we can
successively multiply the $(1-p_{t})$ terms together until the value of
the product falls below u. What Socsim does is mathematically
equivalent to this procedure, however, the implementation in function
\user{datev} takes advantage of fact that the probabilities can be the
same over months or years and works with powers of $(1-p_{t})$.
\section{Heterogeneity multipliers}
\label{sec:hetero}

Heterogeneity beyond that which follows from the algorithm described
in Section~\ref{sec:datev} is
often desirable in microsimulation.  Socsim increases heterogeneity
of fertility for example, in order to create more realistic sibling
set sizes and to allow for heritability of fertility.  

Heterogeneity of mortality and group transitions are not included by
default but much code is in place to allow users to add these features
easily in a way that makes sense for a particular simulation.

The general principle is that if each person at risk of an event is
given a value $\theta$ drawn from a distribution with mean=1 and the
hazard rates used for generating each individual's waiting time for
the event are multiplied by $\theta$, then $E(h\theta)=h$.  Where $h$
is the original hazard of the event.  Thus the overall population's
event history should still reflect the rate structure but with greater
variance/heterogeneity.

\subsection{Fertility multiplier (fmult)}
\label{sec:fmult}

When the heterogeneous fertility option,(\voc{hetfert})  is enabled,
each female in the population is assigned a fertility multiplier.
For the initial population these multipliers may be read in the
\voc{.opop} file or they may be generated by Socsim.  Females born
during the simulation will have multipliers generated by Socsim.  Once
assigned, fertility multipliers remain with the woman for her entire
life increasing or reducing proportionally the hazard of giving birth
at each.

In the current implementation, individual fertility multipliers,
\voc{fmult}s, are pseudorandomly distributed as a cubic approximation to
the beta distribution with mean 1.0, variance 0.416, and a range of 0
to 2.4.  Fertility heterogeneity is also heritable to a degree determined by the user specified directives \keyw{alpha} and \keyw{beta}.

\subsection{Inheritance of Fertility Multipliers}
\label{sec:inheritancefert}

The degree to which heterogeneous fertility is inherited is determined
by the \keyw{alpha} ($\alpha$) and \keyw{beta} ($\beta$) directives.
Equation~\ref{eq:inheritfert} defines the algorithm used to effect the
inheritance. At birth females are first assigned a fertility
multiplier as according to the beta distribution described in
Section~\ref{sec:fmult} this is $\gamma$ in
Equation~\ref{eq:inheritfert}.  The temporary variable $x$ gets the
$\alpha$ weighted average of $\gamma$ and ego's mother's fertility
multiplier. If $\beta=1$ then $x$ is the daughter's fertility multiplier. Otherwise it is further modified by the second equation.

\begin{eqnarray}
  x &=& \alpha *\mathrm{fmult}_{\mathrm{mother}} + (1-\alpha)*\gamma\\ 
\label{eq:inheritfert}
  \mathrm{fmult}_{\mathrm{daughter}} &=& 2.5* \exp^{\beta*log(\frac{x}{2.5})} 
\end{eqnarray}

%      child->fmult = alpha * p->fmult + (1 - alpha) * fertmult();
%      child->fmult = 2.5 * exp(beta*log(child->fmult/2.5));


%%% Local Variables: 
%%% mode: latex
%%% TeX-master: "top"
%%% End: 

% LocalWords:  Socsim microsimulation fmult hetfert opop
