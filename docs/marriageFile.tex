
\section{The marriage file}
\label{sec:marriageFile}
The marriage (.omar) file is similar to the population (.opop) file in
that much of the information it maintains is in the form of unique id
numbers that corespond to rows of the .opop file or to the .omar file
itself.  In other words, the .omar file is another  \emph{linked list}

Figure~\ref{fig:readOmar} shows a snippet of R code suitable for
reading a .omar file.  Table~\ref{tab:marriagefile} shows the meaning
of each column of the file.

\subsection{Reckoning marriages}


As with the .opop file, reckoning marriage histories requires
following a list of integers from one record (row) to another. The
marriage id of the wife's \emph{most recent prior} marriage is stored
in column 7.  The coresponding pointer for the husband is stored in
column 8.  The other six columns of the marriage record hold
information on the marriage itself.  Note that marriages are created
sequentially, but monogamy is not assumed. So for any particular
marriage record, the wife's prior marriage (pointed to in column 7)
must have a start date (column 4) that is no larger (later) than that
of the particular marriage.  A ``prior marriage'' must start before
(or in socsim, at least in the same month) as the subsequent
marriage.  But the same is not true for the marriage end date, which
is stored in column 5.  

Also, be careful of zeros.  Marriages that remained intact when the
simulation ended have end dates of zero.  

To find a female ego's first husband, one begins with the .opop file.
The $11^{th}$ column of ego's row holds the marriage id of egos most
recent marriage. The row in the .omar file whose first column entry
matches that number is the record of egos' most recent or ``last''
marriage.  In order to find her \textbf{first} marriage, we must
locate the row of the .omar file wherein the entry in column 1 matches
the the marriage id of the  wifes prior marriage which is stored in
column 7 of the .omar file.  We repeat this proces until we locate a
record of a marriage for which ego's person id is stored in the second
column and column 7 holds a zero.

In R, a more efficient way of finding first marriages is to select the
subset of marriages for which either the husband's prior or wife's
prior marriage (column 7 or column 8) are zero, and then use the
\user{match()} function to link the marriage id to the husband or
wife's opop record.  Figure~\ref{fig:firstMarriages} shows a snippet
of R code that performs this task.





\begin{figure}[h]
  \centering
\vspace{.25cm}
\rule{.5 \textwidth}{.1mm}
\begin{verbatim}
omar<-read.table(file="../SimResults/example.omar",header=F,as.is=T)
names(omar)<-c("mid","wpid","hpid","dstart","dend",
               "rend","wprior","hprior")
   
rownames(omar)<-omar$mid

\end{verbatim}
  \caption{R code for reading a .omar file}
  label{fig:readOmar}
\end{figure}

\begin{table}[h]
  \centering
  \begin{tabular}{||r|l|p{5cm}||}
\hline
\textbf{position}&\textbf{name}&\textbf{description}\\
\hline \hline
1&mid& Marriage id number (unique sequential integer) \\
2&wpid &Wife's person id\\
3&hpid &Husband's person id \\
4&dstart &Date marriage began \\
5&dend&Date marriage ended or zero if still in force at end of simulation\\
6&rend &Reason marriage ended 2 = divorce; 3 = death of one partner \\
7&wprior &Marriage id of wife's next most recent prior marriage\\
8&hprior&Marriage id of husband's next most recent prior marriage\\
\hline    
  \end{tabular}
  \caption{Structure of Socsim marriage file}
  \label{tab:marriagefile}
\end{table}

%%%%
\begin{figure}[h]
  \centering
\vspace{.25cm}
\rule{.5 \textwidth}{.1mm}
\begin{verbatim}

## get first marriage id -- socsim stores marriage ids as linked list
## headed by most recent marriage.  The (h/w)prior field stores the id
## of each spouses prior marriage

## select a subset of marriages which are first for at least one partner
fomar<-omar[omar$hprior == 0 | omar$wprior == 0,]
opop$fmid<-NA

## use match() to lookup the marriage id of each person (in opop)'s
## first marriage id
opop[opop$fem==0,"fmid"]<-
  fomar[match(opop[opop$fem==0,"pid"],fomar$hpid),"mid"]
opop[opop$fem==1,"fmid"]<-
  fomar[match(opop[opop$fem==1,"pid"],fomar$wpid),"mid"]

\end{verbatim}
  \caption{R code for finding first marriages}
  label{fig:firstMarriages}
\end{figure}


%%% Local Variables: 
%%% mode: latex
%%% TeX-master: "top"
%%% End: 

% LocalWords:  Socsim Socsim's
