
\section{Specifying demographic rates}
\label{sec:rateFiles}

It is most convenient to store demographic rates for each simulation
segment in distinct files and use the \keyw{include} directive to
reference them from the supervisory file (See
Section~\ref{sec:supFile}).  But regardless of how you choose to organize your rate files,  you will need to assemble a large collection of rates. How Socsim expects those rates to be formatted is described in Section~\ref{sec:vitalRates}, but before we get to that Section~\ref{sec:rateDefault} describes the rules Socsim follows when it encounters incomplete rate sets.  These are important to understand, because Socsim does not warn you when for example you leave out fertility rates for parity 1 divorced females of group 3  (birth 3 F divorced 1).  Instead, it ``defaults'' to rates for party 0 divorced females of group 3 (birth 3 F divorced 0). Similarly if the birth 3 F divorced 0 rates are missing, Socsim uses birth 3 F single 0) in their place.  The rules are fairly intuitive, but its important to understand that one can err by not specifying that certain rates are zero.  

\subsection{Rate default rules}
\label{sec:rateDefault}


To run a moderately realistic simulation, Socsim requires age specific
fertility rates for females and mortality rates for both males and
females of each \voc{group} and marital status.  If rates do not
differ by marital status or group then you can use Socsim's default
rules to avoid entering the same blocks of rates repeatedly.  When Socsim
encounters incomplete rates sets, it follows a set of rules to determine how the blanks are to be filled in.
Figure~\ref{fig:rateDefaults} shows the rules that Socsim uses when it
encounters incomplete rate sets. The ``==>'' symbol in
Figure~\ref{fig:rateDefaults} means ``defaults to'' so for example,
\begin{verbatim}

widowed                         ==> divorced; parity 0; group 1

\end{verbatim}
which appears in the ``Fertility Rates'' section under the heading
``For parity zero women in group 1'' 
indicates that if Socsim does not find fertility rates for parity zero, widowed
females in group 1 it will ``default to'' the rates for divorced women of parity zero and group 1.


 Where Figure~\ref{fig:rateDefaults} indicates that a rate block
 defaults to ``Zero'', Socsim does not default to anything leaving
 such events with zero probability of occurring at any age. So for
 example, unless you think that single males in group 1 should live
 forever, you must specify mortality rates for such ``people''.
 \clearpage
\begin{figure}
  \label{fig:rateDefaults}
  \caption{Rate default rules}

\center{\rule{.5\textwidth}{.1mm}}\\
\textbf{Fertility Rates}

\begin{verbatim}
    
For parity zero women in group 1 
--------------------------------
single                          ==> Zero
married                         ==> Zero
divorced                        ==> single; parity 0; group 1
widowed                         ==> divorced; parity 0; group 1
cohabiting                      ==>  married; parity 0; group 1

For women in group 1 with higher parity
------------------------------------------
mstatus m; PARITY P; group 1     ==> mstatus m; PARITY P-1; group 1

For women of any parity and any group > 1

mstatus m; parity  p; GROUP  G   ==>  mstatus m; parity p; GROUP G-1

\end{verbatim}

\textbf{Marriage, Divorce and Mortality Rates}
\begin{verbatim}

For men and women in group
-----------------------------------
DEATH    for single; sex s; group 1       ==> Zero
MARRIAGE for single; sex s; group 1       ==> Zero
DIVORCE  for mstatus m; sex s; group 1    ==> Zero

  (Events except for divorce)
-----------------------------
event e for divorced;sex s    ==>  event e; for single;sex s;group 1  
event e for widowed;sex s     ==>  event e; for divorced;sex s;group 1  
event e for married;sex s     ==>  event e; for widowed;sex s;group 1  
event e for cohabitting;sex s ==>   event e; for married;sex s;group 1  

For Groups > 1
--------------
event e for mstatus m;sex s; group g ==> event e for mstatus m;sex s; group g-1

\end{verbatim}
\textbf{Group Transition Rates}
\begin{verbatim}
TRANSITTION to group H for mstatus m;sex s; group g == Zero
\end{verbatim}

\center{\rule{.5\textwidth}{.1mm}}\\
\end{figure}
\subsection{Structure of vital rates}
\label{sec:vitalRates}

The format in which Socsim expects to find rates is simple. Each
\voc{block} of rates begins with a set of keywords which indicate the
event for which the rates apply and the marital status, sex, group
membership and possibly parity of the people who are at risk of
experiencing the event.  That information is followed on subsequent
lines of rate values and the \textbf{upper bound} of the age group for which
the rate is in force.  

A \voc{rate block} is a complete set of age specific rates governing a
demographic event for people of a particular sex,group and marital
status. An example of a rate block is shown in
Figure~\ref{fig:rateBlock}.  The first line after the comment line,
indicates which event (death); group (1); sex (M=male); and marital
status (single) this rate block pertains to. The order matters and is
always, Event then group then sex then marital status. In the case of
birth rates this may be followed by number indicating parity. In the
case of transition rates, the line must end with a number indicating
the destination group.

Each subsequent line contains a one month rate (in the case of
fertility) or a one month probability in the case of all other events,
and the age interval over which the monthly rate (probability) holds.  The
first two numbers in the line are years and months of the upper age
bound.  These are added together so a 1 and 11 would mean 23
months. The third number is the rate. In the case of fertility it
represents the expected number of births \textbf{per month} to a
woman who survives to end of the given age interval. Specifically -- The interval that includes  upper age bound given in the previous line and ends just before the upper age bound given on the \textbf{current line}.


Figure~\ref{fig:rateBlock} shows an example of rate block. The first line indicates that that rates which follow refer to mortality  of group 1 single males (death 1 M single).

The first first rate line (\verb|0   1   .0460940|) indicates that the probability of death in the first month of life for males (technically for never married males) is .0460940.  Note that a rate line with an upper age bound of ``0 0'' is meaningless and is ignored.  So be careful when specifying infant mortality rates.
 

Taking another line from Figure~\ref{fig:rateBlock}, the probability that a single male dies
between the ages of 1141 and 1200 months, conditional on having
survived to the beginning of the age interval is $1-(1-.08326)^{60}$.
\begin{figure*}[h]
  \centering
\label{fig:rateBlock}
\caption{Example of a block of mortality rates}
\begin{verbatim}
*Mortality, single Male (lines beginning with * are comments)
death 1 M single
0       1       .0460940
0       12      .0057540
0       60      .0008730
0       120     .0002600
.
.
.
0       1140    .0832630
0       1200    .0832630

\end{verbatim}
\end{figure*}


\subsection{Mortality rates}
\label{sec:mortRates}

Mortality rates are the most straight forward of the rates that Socsim
uses. The example in Figure~\ref{fig:rateBlock} is typical. The event
identifier is \keyw{death}, the ``1'' refers to the group, ``M'' to
male and ``single'' is of course marital status.

\subsection{Marriage rates}
\label{sec:marriageRates}

Marriage rates are specified with the event identifier of
\keyw{marriage}, but it should be born in mind the event which these
rates regulate is not really marriage but rather the commencement of a
marriage search.  Since marriage requires two participants Socsim
cannot simply execute a marriage when the event is scheduled.  Marriage
requires two participants and such it is very difficult to achieve arbitrarily specified marriage rates for both males and females. If a marriage age distribution is also part of the simulation, it gets even harder.  Socsim deals with this in a variety of ways which are described in Section~\ref{sec:marriageQueue} and
Section~\ref{sec:supFile}.

As a consequence, of all this excuse making is that marriage rates
often need to be ``tuned'' in order to achieve the desired result.

\subsection{Divorce rates}

Divorce rates are specified with the event identifier \keyw{divorce}.
Divorce is unusual among Socsim events in that it's rates do not apply
to the age of one spouse or the other but rather by age of the
marriage.  It is thus generally not necessary to specify divorce rates
for both sexes.

\subsection{Fertility rates}

Fertility rates are specified with the identifier \keyw{birth} and are
different from other rates in two ways:
\begin{enumerate}
\item They are parity specific. But they default to parity $n-1$ so it
  is only necessary to specify rates for parity zero.
\item They are rates rather than probabilities. So multiplying a
  rate by the number of months in the age category gives the expected
  number of births that a woman who lives through the age category
  will experience.
\end{enumerate}

\subsection{Transition rates}
\label{sec:transit}
Transition rates give rates of ``transition'' from one \voc{group} to
another.  By default, no transitions occur, however, if the initial
population contains more than one group then the group inheritance
rule determines the group identity of new borns.

Transition rates are specified with the identifier \keyw{transit} and
are different from other rates in that both the group to which the
rate applies \textbf{and} the group to which the event will cause a
person to belong must be specified.  

To specify transition rates from group 1 to group 2 for single males,
one would write the following:\\
\\
trasit 1 M single 2\\
\\
As noted in   Figure~\ref{fig:rateDefaults}, transition rates have no
defaults. All rates  have to be specified in order to take effect. 

\subsubsection{Duration specific transition rates}
\label{sec:duratspectrans}

It is often useful for transition rates to be duration rather than age
specific. In other words, the probability of a transition event
occurring can depend on the time since the individual transitioned
into the current group rather than the individual's age.  In order for
this distinction to matter, a person must have experienced at least
one transition other wise the time spent in the group is equivalent to
the person's age.

Divorce rates are always duration specific, but transition rates may
vary with either age or duration.  The format for specifying
transition rate blocks is the same regardless of whether they are age
or duration specific. To tell Socsim that a particular rate block is
meant to be
duration specific, add the directive: \keyw{duration\_specific} as in:\\
\\
duration\_specific transit 4 F married 5\\
\\
when this directive is encountered, the log file will indicate that
transition rates from group 4 to group 5 for married females will be
duration rather than age specific.  The default is for transition
rates to be age specific so no indication is given in the log file of
that condition.  It is possible to have both age and duration specific
transition rates in effect in the same simulation, however, only one
transition rate block is allowed per pair of groups.  So transitions
from say group 3 to group 5 for single males must be \emph{either} age
or duration specific, but whichever it is, has no effect on
transitions from group 3 to group 5 for \textbf{\emph{married}}
males. One may be age specific while the other may be duration
specific.

As noted in Section~\ref{sec:segmentSpecific}, the
\keyw{duration\_specific} directive does not replace the keywords
that define the rate block.

  


%%% Local Variables: 
%%% mode: latex
%%% TeX-master: "top"
%%% End: 

% LocalWords:  Socsim
