
\section{Population  files}
\label{sec:populationFile}

Socsim reads and writes everything it knows about people in two files:
the population  \voc{.opop} file and the marriage \voc{.omar}
file\footnote{It may also read and write a file of extra variables and
  a file of transition history}.  Both are
space delimited files and both contain only numbers. 

Since one generally runs Socsim for 200 simulated years in order to
start from a population with a known and stable age structure, it is
seldom necessary to construct an initial population with any
information other than the age and sex of a small number of
individuals. Such a file can be easily constructed in a spreadsheet
program.  The coresponding marriage file is simply an empty file with
correct name.

One needs to come to terms with the structure of the .opop and .omar
files in much more detail when analyzing simulation output.  A snipet
of code for reading an .opop file into R is given in
Figure~\ref{fig:readOpop}.  In terms of R the .opop file's contents
fit naturally into a \voc{data.frame} with 14 columns all of which are
numerical. In more general terms, the .opop file is a matrix where
each row contains information on a single person and each column
contains a particular bit of infromation on each person. 

Although it's structure suggests that a .opop file might be right at
home in a spreadsheet program, this is not so.  First the files tend
to be too large since they include not only a row for each person who
ever lived. Even a modest sized simulation can easily have 100,000
rows.  But more important, much of the information in the .opop file
consists of identification numbers of other people.  In other words
the opop file is a multiply linked list.  For manipulating linked
lists, spreadsheets are profoundly suboptimal.

\begin{figure}[h]
  \centering
\vspace{.25cm}
\rule{.5\textwidth}{.1mm}
\begin{verbatim}
## read .opop into dataframe the .opop file contains one row for each
## simulated person who ever "lived". It generally includes many who
## "died" long ago.

opop<-read.table(file="../SimResults/example.opop",header=F,as.is=T)

## assign names to columns
names(opop)<-c("pid","fem","group",
               "nev","dob","mom","pop","nesibm","nesibp",
               "lborn","marid","mstat","dod","fmult")


\end{verbatim}
  \caption{Rcode for reading simulation output files}
\rule{.5\textwidth}{.1mm}
  \label{fig:readOpop}

\end{figure}


In most cases, the .opop file will be sorted in order of person id and
since person ids are sequential integers assigned in birth order this
means that the .opop file is generally sorted in birth order, with the
row number often being the same as the person id. This is not
however guaranteed to be the case so check that it is so before
relying on it\footnote{It is best practice not to rely on the .opop
  file's pid ordering because this convention can be broken in the
  initial population file}.

Table~\ref{tab:opop} shows which information is in each column.
\begin{table}[h]
  \centering

  \begin{tabular*}{.8\textwidth}{||r| l |p{6cm} ||}
\hline
\textbf{position}&\textbf{name}&\textbf{description}\\
\hline \hline
1&pid & Person id unique identifier assigned as integer in birth order\\
2&fem & 1 if female 0 if male\\
3&group& Group identifier 1..60 current group membership of individual\\
4&nev & Next scheduled event \\
5&dob & Date of birth integer month number\\
6&mom & Person id of mother \\
7&pop & Person id  of father \\
8&nesibm & Person id of next eldest sibling through mother\\
9&nesibp & Person id  of next eldest sibling through father\\
10&lborn & Person id  of last born child\\
11&marid & Id of marriage in .omar file \\
12&mstat & Marital status at end of simulation integer 1=single;2=divorced;
          3=widowed; 4=married\\
13&dod & Date of death or 0 if alive at end of simulation\\
14&fmult & Fertility multiplier\\
    \hline
  \end{tabular*}
  \caption{contents and format of the .opop file}
  \label{tab:opop}
\end{table}

\subsection{Reckoning kinship}
In analyzing Socsim output, one is often interested in reckonning
kinship.  Since with the possible exception of the initial population,
everyone in socsim is related to everyone else, it is possible to find
nearly any two people's relationship by following the chain of parents
and siblings. To find ego's maternal grandmother, one simply finds the
ego's mother's person id in the $6^{th}$ column of ego's row in the
opop file. Moving then to the row of the opop file coresponding to
ego's mother's person id, one look's again in the $6^{th}$ column to
find egos' mother's mother's person id.

To find all of ego's children, one starts with the person id of ego's
last born child (stored in column 10) of egos' row of opop.  In the
row coresponding to ego's last born child's row of opop, we find, in
column 8 (9), ego's last born child's next eldest sibling through her
mother (father).  In that person's row of the .opop file, we can find
yet another next eldest sibling and so on until we find ego's first
born child, whose next eldest sibling through mother (assuming that
ego is female) is necessarily zero.

Alternatively, one could simply collect all the rows of opop which
have that same value in column 6 (mom) and or 7(dad) as ego has.

The R computing environment (\url{http://www.r-project.org}is
particularly well suited for doing this kind of analysis of kinship.

\subsection{Reference to  marriages}

Since individuals can be married more than once (simultaneously in
some cases) reckoning marriage information is trickier than working
with kinship alone.  See Section~\ref{marriageFile} for more details
on how to work with Socsim's .omar file.  For the present purpose note
that column 11 of the .opop file contains a pointer, in the form of a
marriage id number, to ego's most recent marriage.  If column 11 is
zero, then ego has never been married.

\subsection{Reference to transition history}
If the simulation includes group transitions, than socsim will write a
file with the same path as the population and marriage files but with
the suffix \voc{.otx}.  The transition events do not have unique ids
as do marriages, but each transition record contains the person id of
the protagonist. Consequently it is much more natural to link from the
transition history to the population file.

Section~\ref{otxFile} describes the \voc{otx}  file in detail.


%%% Local Variables: 
%%% mode: latex
%%% TeX-master: "top"
%%% End: 
