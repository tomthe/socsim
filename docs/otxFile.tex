
\section{Transition history files}

If group transitions are part of the simulation then socsim will write
(and may read) a transition history file.  In spirit, the \voc{otx}
file is much like the marriage file, the main differences are that the
\voc{file} is linked to only one person and there is no unique
identifier for a transition history record.

Like all other Socsim output files, the \voc{otx} file is space
delimited and contains only numbers. Table~\ref{tab:otxformat} shows
the structure of the \voc{otx} file. Figure~\ref{fig:otxread} contains
R code for reading an otx file into a data frame.



  \begin{table}[h]
    \centering
    \begin{tabular*}{.8\textwidth}{||r|l|p{6cm} ||}
      \hline
      \textbf{position}&\textbf{name}&\textbf{description}\\
\hline \hline
      1&pid& Person id to who the transition event occurred\\
      2&date&Month in which the transtion occurred\\
      3&fromg&group from which the person transitions\\
      4&tog&group into which the person transitions\\
      5&sequence& a non positive number indicating the order of the
      event. A zero indicates that the current record refers to the
      most recent transition event; a -7 
      indicates that seven transitions have occurred to this person
      subsequent to that of the current record.\\
      \hline
    \end{tabular*}
    \caption{Contents and format of the .otx file}
    \label{tab:otxformat}
  \end{table}




  \begin{figure}[h]
    \centering
\vspace{.25cm}
\rule{.5\textwidth}{.1mm}

\begin{verbatim}
otx<-read.table(file="../SimResults/test.otx",header=F,as.is=T)
names(otx)<-c("pid","month","fromg","tog","pnum")

\end{verbatim}
    \caption{R code for reading an  \voc{otx} file}
    \label{fig:otxread}
  \end{figure}

%%% Local Variables: 
%%% mode: latex
%%% TeX-master: "top"
%%% End: 
