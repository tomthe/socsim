
\section{First Encounter}
\label{sec:firstEncounter}



Originally written in the 1970s, Socsim continues to develop as a research tool
constantly changing to meet the goals of new research projects.  A slick
graphic user interface has never been part of the plan\footnote{to the
  extent that there is a ``plan''}. As a consequence, users must be
comfortable editing ascii files in order to use socsim.

Raising the barrier to first use still higher is Socsim's insatiable
appetite for input data. Because micro-simulation is about simulating a
whole mess of ``individuals''' vital demographic events, Socsim needs a
whole mess of age, sex, marital status, \voc{group}\footnote{Groups
  are a additional category that can be used to extend Socsim in
  interesting ways. Groups are described in a little more detail in Section~\ref{sec:groups}} and parity specific rates
for births, deaths, marriages and change of group membership. There is
of course, a default structure -- for example if mortality rates are
not specified for married males, Socsim will by default use those of
single males of the same group.  Or if parity specific fertility rates
are not specified, Socsim will use the same rates for all
parities. Still, a moderately complicated simulation will still require
an \emph{astonishing} number of rates.

\subsection{The simplest possible simulation}
\label{sec:simplest}

To run Socsim you must provide the following:
\begin{description}

\item[A compiled version of Socsim.] See \url{lab.demog.berkeley.edu/socsim} if
  this has not already been taken care of.
  
\item [An initial population and marriage file.] Socsim reads and
  writes Population data in a specific format. Population (or
  \voc{.opop} files) are further described in
  Section~\ref{sec:populationFile}.  In order for Socsim to start, it
  is must find an \voc{initial population} stored in this format.
  Generally individuals in this population will not be married, if the
  initial population has a marriage structure, then a
  marriage file (or \voc{.omar}) will also be required. The marriage
  file format is described in Section~\ref{tab:marriagefile}.  
  The initial population need have no particular age or sex
  distribution\footnote{ One \textbf{can} start Socsim with an initial
    population complete with a kin network, but typically the purpose
    of this whole exercise is to generate a simulated a kinship
    network.%% -- if you already have the kinship data then maybe you
    %%should stop wasting time and write your dissertation
}, since one usually
  runs Socsim for a  hundred (simulated) years or more in order to
  convert the initial population into a population with a known and
  stable age structure before simulating the processes that are of
  real interest.  This can be done differently, but not without added
  complication.

\item [A \voc{supervisory or .sup file}.] The supervisory file
  generally named with \user{.sup} suffix is passed to Socsim on the
  command line. It must contain certain \voc{global parameters}  such as the
 number of simulation
  \voc{segments}, name patterns of initial population and final
  population files, and either a complete set of vital
  demographic rates or else \voc{include} statements specifying where
  to find files containing those rates.  The full explanation of
  \voc{.sup} file is in Section~\ref{sec:supFile}.

The minimal file, (from the \user{Sample} directory of the source code
tree) is shown in Figure~\ref{fig:supSample}.


\item[vital demographic rates]  As noted previously, the vital
  demographic rates can be included within the supervisory file. But
  generally it is more convenient to store them in one or more
  separate files which are referred to in the supervisory file with an
  \user{include} directive. 

The format in which the rates are given
  to Socsim is critical. Generally a set of rates for particular
  demographic event begins with an identifier such as:\\
\\
\textbf{birth 1 F married 0 }\\
\\
which indicates that the following several lines will contain a
complete set of age specific fertility rates for group 1 married
females of parity 0. 

Following the rate identifier will be one or more lines each of which
contains three space-separated numbers. The first two indicate year and
months of the \textbf{upper age bound} over which the rate is in
effect. The two quantities are added together so ``1 1'' is equivalent
in effect to ``0 13''.  Note that the first age category specified
should NOT have ``0 0'' as its upper bound; an upper age bound of ``0
1'' would include the first month of life only.  Age categories are
otherwise flexible;  one can specify some age categories that are as
short as a single month, while others are many years or decades wide.

The rate, given in per month terms, is the third number. Socsim
insists that there be a line specifying an age category with an upper
bound of 100 years. Obviously this last rate can be 1 or zero.  For
more details on specifying rates See Section~\ref{sec:rateFiles}.
\end{description}

When all of the above bits are in place, Socsim can be executed from
the Unix shell:\\

\user{ /path/to/socsim supervisoryfile.sup 12345}\\

The result will be a considerable amount of screen output indicating
which options are set and how the simulation is proceeding. Or just a
short complaint about an inconsistency in the various input files.

After a successful socsim run, a logfile called
/user{supervisoryfile.sup12345.log} should be evident in the current
directory as well as an output population \fname{.opop}; marriage,
\fname{.omar} and perhaps a extra variables, \fname{.opox} file whose
locations depend on the \user{output\_file} directive in the
\fname{.sup} file.  Socsim also write an event history file


The \fname{.opop}, \fname{.omar} and \fname{.opox} files produced by a
Socsim run, are of course suitable initial population files(s) for
subsequent Socsim simulations.  Eventually however, you will probably
have to stop simulating and start analyzing those files.  See
Sections~\ref{sec:populationFile}and \ref{sec:marriageFile} for
details on how these files are structured.

\subsection{What goes on inside}
\label{sec:inside}

While all that ``information'' is scrolling past on the screen and
filling the log file, Socsim is busily scheduling and executing vital
demographic events for each simulated being in the initial population
as well as all of their descendants.

Socsim begins each simulation \voc{segment}\footnote{A simulation
  segment is a period of simulated time during which a single set of
  vital rates are in effect} by scheduling an event for each
``living'' person.   At all times during the simulation every non dead
simulated being has exactly one scheduled event.  Whenever an event is
executed or a person's marital status or parity changes, a new event
is scheduled for that individual.  Thus death of a spouse causes a new
event to be scheduled for the widow; a birth causes a new event to be
scheduled for both the child and the mother.

To determine which event a person is to be scheduled for, Socsim
generates a random waiting time for each event that the person is at
risk of having.  Obviously the user supplied age, sex, group, and
marital status specific rates govern the process of random waiting
time generation.  Once all of the potential events have randomly
generated waiting times associated with them, the earliest one is
chosen and that event is scheduled.  Information on the other possible
events is discarded.  In short, it's a competing risks model.  The
probability of each event is independent of all other events and the
earliest event is the only one that counts. See
Section~\ref{sec:eventcomp} for more details.

Once each person has a scheduled event, Socsim starts to march through
time. A list is created of all of the events scheduled for the first
month of the simulation segment and each is drawn in random order and
executed.  When the execution of an event causes a new event to be
scheduled (as most do) the resulting new event is either placed on the
calendar for execution in a future month or if the waiting time is
zero, it is  inserted in the list of events waiting for
execution in the current month and is drawn in random order.

When all of the events scheduled for the current month are executed,
Socsim increments the month and repeats the event execution procedure
just out lined.  When the last month of the segment is completed,
Socsim begins the next segment by reading the rate files and
generating new events for all living people.  If there is no
subsequent segment, then Socsim finishes by writing the population
(.opop), marriage (.oar), extra variables (.opox), and population
pyramid (.pyr) files.

\section{Extending Socsim}
\label{sec:extending}

Socsim is open source so modifying the code to take account of
cultural norms, ethnological truths, or behavioral theories is
encouraged.  Since many Socsim based research projects require some
kind of enhancement, the program is structured so as to encourage
modification. It is possible to enhance Socsim without first gaining a
deep knowledge of all the intricacies of the program by taking
advantage of ``hooks'' that have been placed in the code and which
call functions at key points in the execution.  ``Stub'' versions of these
functions can be found in the \user{enhancementNULL.c} file in the top
of the source tree.  

When Socsim is compiled with the proper command line arguments,  functions in
\user{enhancement.c} become part of the code.  If your modifications
can be contained within \user{enhancement.c}, then improvements and
bug fixes to the main trunk of the code will not conflict with your
project -- probably.

Before you consider enhancing Socsim, however, you should first give a
lot of serious thought to how you might manage to accomplish your goal
\textbf{without writing any C code}.  One approach for doing this is
is to try to make clever use of the concept of \voc{groups}, described
below.

Another approach to email Carl Mason (carlm@demog.berkeley.edu) and
see if maybe he'll do the programming for you.
\subsection{Groups}
\label{sec:groups}

 Socsim implements  \voc{groups} by simply adding a variable to the
 population list. Every person is a member of exactly one group at all
 times. Group membership is determined at birth according to the rule
 given in  \voc{child\_inherits\_group} directive, but can change at
 any time according to age, sex, marital status, and group specific
 \voc{transition} rates that the user specifies.  

 The group designation can be thought of as ethnicity, location,
 health status, wealth, education or any characteristic of a person
 that might or might not change according to a rate schedule.

 Members of each group are subject to the same rates for vital
 events. That is the members of group 7 have fertility, mortality and
 nuptiality schedules that are independent of those of group 8.  When
 an individual transitions from one group to another, she becomes
 subject to the new group's rates and therefore must have a new event
 scheduled. 


 Socsim determines how many groups are in a simulation at the
 beginning of each segment by looking at both the starting population
 and the rates.  If Socsim finds a person of group N in the initial
 population than it knows that there are at least N groups in the
 simulation. If it finds a mortality rate for members of group Q then
 it knows that there are at least Q groups.

Because of the default pattern of rates described in
Section~\ref{sec:rateFiles}, it is possible to add groups to a
simulation and only specify the rates for the group which differ from
the those of the existing group.


%%% Local Variables: 
%%% mode: latex
%%% TeX-master: "top"
%%% End: 

% LocalWords:  Socsim Socsim's opop
