
\section{Marriage Market}
\label{sec:marriageQueue}

The marriage event is greatly complicated by the need for two spouses.
When a 27 year old female's marriage event comes up, there must be a
suitable male at hand\footnote{Socsim, does not yet handle same sex
  marriage. Sorry} or else that marriage cannot be executed on
time. The need for two spouses means that randomness alone is
inadequate for assuring that the specified marriage rates are
achieved.

More generally, one can meet \emph{at most} two of the following three
marriage market constraints:

\begin{itemize}
\item Female age specific nuptiality rates
\item Male age specific nuptiality rates
\item Distribution of spousal age differences
\end{itemize}

Thus compromise is necessary and Socsim offers a choice of two: A
``one'' queue and a ``two'' queue system.  In the \emph{two queue} system
male and female nuptiality rates determine the beginning of a marriage
search: only egos who have initiated marriage searches can be married
under this system. If no suitable potential spouses are available, the
searcher must then wait in a queue. Because a suitable spouse must be
waiting in order for a marriage event to be executed, generally one of
the two spouses must be getting married \emph{after} a delay of some
time since the beginning of her search.  

In the \emph{one queue} system, only females have nuptiality rates.
When a marriage event is executed under this system, the lucky bride
chooses the best match from \emph{all} living unmarried males.  Thus
under the one queue system, female nuptiality rates are generally met,
and marriage markets can be made to optimize selections over even rare
criteria.  In the unmodified version of Socsim, only the distribution
of the age difference between spouses is considered, but in principle
the scoring algorithm could be made to optimize over a set of
criteria. 

The cost of this however are (1) no male marriage rates can
be specified -- males in the \emph{one queue} scheme marry at the
convenience of females -- and (2) processing will be slower especially when
populations are large.


\subsection{The two queue system}

Under the legacy ``two queue'' scheme, the marriage event signals the
beginning of a marriage \emph{search}. If a suitable spouse is
\emph{already} waiting on the marriage queue, then the marriage is
executed. If not, then ego's marriage search fails and she is added to
the marriage queue herself--where she will wait (subject to the risk
of other events) until she is ``selected'' by a male suitor who is
initiating a marriage search.  Under this scheme, the specified
nuptiality rates are seldom accomplished because would be spouses
often spend time on the marriage queue.  This scheme is symmetric with
respect to sex. There is a marriage queue for each sex and each sex is
treated identically.

If, when a marriage search is initiated, there are potential spouses
waiting in the queue then each potential match is fist checked for
allowability via the \user{marriage\_allowable()} function and, if the
match does not violate incest rules and the potential spouses have not
been previously married, then the match is evaluated by the
\user{score3()} function and the highest scoring match is executed.


\subsection{The one queue system}
\label{sec:oneq}

The second (newer) scheme is a ``one queue'' system under which the female
marriage event triggers a search across \textbf{all} available males
and the ``best'' one is immediately selected. Potential matches are
evaluated in the same way as under the two queue system (via the
\user{score3()} function), but many
more potential matches are evaluated for each event.

The main advantage of the one queue system is that it generally
achieves the specified female age specific marriage rates --even when
female rates are high and populations are small. Achieving the female
marriage rates allows for overall fertility rates to be achieved when
marital and non marital fertility rates are different. This is major
headache preventer.

A second advantage of the one queue system is that it allows for a
wider range of marriage criteria to be optimized. This of course requires
that Socsim be \emph{extended} (See Section~\ref{sec:extending}), but
since under the one queue system \emph{every} marriageable male is
examined for each female marriage event, even spouses with qualities
that are quite rare in the population can be located.  Under the two
queue scheme one can only choose the best spouse from those who have
already initiated but not completed a marriage search. This can be
quite unsatisfactory if for example the simulation is of a marriage
market in which the optimal spouse is a particular type of cousin.


\subsection{Evaluating potential marriages (score3())}
\label{sec:score3}

Regardless of which marriage queue system is enabled, potential
marriages are fist screened by the \user{marriage\_allowable()} and
then scored by the \user{score()} function. 


The \user{marriage\_allowable()}function, eliminates consanguineous
marriages closer than cousins and remarriage of spouses who have
previously been married to each other. 

Potential allowable marriages are then evaluated by the
\user{score3()} function.  There are two variants built into Socsim,
hooks are included in the code to make modification of the evaluation
process ``easy''.

\subsubsection{Age distribution matching}
\label{sec:agedist}

If \keyw{marriage\_eval} is set to ``distribution'' then Socsim seeks
to select the marriage that most reduces the disparity between the
\voc{target distribution of age differences at marriage} and the
\voc{observed distribution of age differences at marriage}.  The
\user{score3()} function does this by keeping track of the spousal
ages at marriage of all marriages in the simulation segment as well as
a predetermined target distribution. Each potential marriage is
assigned a score that indicates the degree to which the fit between
the target and observed distribution would be changed by the marriage
in question.  The marriage that most reduces the unweighted sum of the
differences between the fractions of marriages conducted at each
spousal age difference (in single years) is then chosen.

\textbf{This variant is intended to be used with the one-queue} system
in which all living unmarried males are potential partners.  In this
situation then, Only female rates and the parameters of the target
spousal age distribution determine marriage events.  In other words,
the age distribution matching scheme together with the ``one queue''
marriage market attempts to optimize the distribution of spousal age differences
subject to the constraint that females marry according to their rate
schedule.

When the age distribution matching scheme is paired with the two-queue
marriage market, the constraint set is much more complicated and the
results are uncertain.



Mechanically, the age distribution matching scheme is implemented in
the following way:

\begin{itemize}
\item At the beginning of each simulation segment, a vector of
  ``target'' \emph{proportions} of age distributions is constructed
  for each simulation group. This vector is derived from the normal
  distribution with parameters: \keyw{agedif\_marriage\_mean} and
  \keyw{agedif\_marriage\_sd}. It  sums to 1 and represents that desired
  fraction of marriages to \textbf{women} of that group that should
  have the particular age difference at the time of marriage.

\item The marriage queue consists of a linked list of all males(females) who
  are eligible. Under the one-queue system this would be all males who
  are not presently married. Under the two-queue system, it
  consists of males (females) who have had unsuccessful marriage events--a
  marriage event triggered a marriage search but no suitable spouse
  was available at that time.  

\item Note again that this marriage evaluation scheme is intended to
  be used with the one-queue marriage market under which each male is
  put on the queue at birth and upon execution of any event that
  subsequently makes him marriage eligible (e.g. divorce or death of
  spouse).

\item When \textbf{under the one-queue system}: a female has a
  marriage event, or \textbf{under the two-queue}: system an ego of
  \emph{either} sex initiates a marriage search, each potential
  partnership is given a score based on the age difference between the
  potential groom and bride.  That score is derived from
  Equation~\ref{eq:mscore}.  Note that the score can be either
  positive or negative. The match with the highest score is chosen. If
  more than one match has the same score, one is chosen at random.
  The result (intention) is that the match which most reduces the
  difference between the target and observed distribution of age
  differences at marriage is the one selected.

  Socsim can be extended to use additional information in this scoring
  algorithm.


  \begin{eqnarray}
    \mathrm{agediff} &=& \mathrm{age}_{\mathrm{groom}} - \mathrm{age}_{\mathrm{bride}} \\    \label{eq:mscore}
    \mathrm{score} &=& \mathrm{target}_{\mathrm{agediff}} - \frac{\mathrm{count}_{\mathrm{agediff}}}{\mathrm{count}_{\mathrm{total}}}
  \end{eqnarray}


\item After each marriage is executed, the counts of marriages by
  female group and age difference is incremented.

\item At the end of the simulation, the accrued differences between the target
  and actual marriage age differences are reported.

\end{itemize}

\subsubsection{In practice ...}
\label{sec:inpractice}

While the 'distribution' system of marriage evaluation can in theory
match both the female age specific marriage rates \emph{and} the
specified distribution of age differences at marriage, ``reality'' has
many ways of defeating the scientist:

\begin{description}
\item[Remarriage] of widows and divorcees: In high mortality
  populations where widows remarry, this -- particularly old widows,
  the average age difference at marriage can wind up being too low and
  the shape of the age difference distribution can diverge from
  normal.

\item[Large mean age differences are difficult] to achieve.  If
  birth cohorts are roughly equal in size it is difficult to sustain a
  symmetric distribution of age difference at marriage when the mean is
  far from zero--unless a large fraction of people never marry.  

\item[Beware of exogamy] If marriages happen between members of
  different groups and the different groups have significantly
  different vital rates, then a lot of strange things might happen.
  While Socsim does manage the age difference distribution separately
  for each group -- it does so from a female perspective. That is the
  marriage age difference distribution for group N refers to the
  distribution of age differences at marriage (groom -bride) for
  marriages involving a \emph{female of group N}. The male's group
  membership is not considered.
\end{description}


\subsubsection{Age preference}
\label{sec:agepref}
If \keyw{marriage\_eval} is set to ``preference'', then \user{score3()}
implements the ``simple'' age preference schedule. In this system,
potential marriage are evaluated according to a fixed preference
scheme. Marriages are preferred according to how close the spouses are
in age to the \keyw{marriage\_peak\_age} no attention is paid to the
realized distribution of marriages.

After incest and endogamy/exogamy checks are complete, suitors who
have not been eliminated, are assigned a score based on the groom's
age - bride's age.  If this age difference is greater than
\keyw{marriage\_peak\_age} than the score declines at a constant rate
as the difference increases.  If the age difference is \textbf{less}
than \keyw{marriage\_peak\_age} then the score declines at
\keyw{marriage\_slope\_ratio}* that constant rate as the age
difference decreases.

Before giving yourself a bald spot over this, note that only the
ordinal rank of this score matters because the spouse is chosen
randomly only from among the maximum scoring suitors.  The idea is
simply to favor a certain optimal age difference and to allow for a
preference for older or younger spouses.  In most simulations the
marriage rates -- which determine the age at which people initiate a
marriage search, are \textbf{far} more important in determining
marriage patterns than are these marriage preference parameters. The
default value for \keyw{marriage\_peak\_age} is 36
months. 

The \keyw{marriage\_slope\_ratio} parameter determines the degree to
which groom-older marriages are preferred to bride-older marriages. A
value of 1.0 implies an equal preference score for an age difference
that is x months from \keyw{marriage\_peak\_age} regardless of whether
the groom or bride is older.  A value greater than 1 implies a that
the preference score will decline faster as the brides age increases
relative the the grooms -- than it will as the groom's age increases
relative to the bride's.  The default value, 2, means that preference
scores decline twice as fast with distance from
\keyw{marriage\_peak\_age} as the bride's age increases relative the
the groom's.

The \keyw{marriage\_agedif\_max} and \keyw{marriage\_agedif\_min}
simply exclude from consideration marriages where the age difference
between the partners (groom age - bride age) is beyond the specified
bounds.

\textbf{While it is possible to use this evaluation scheme
  with the one-queue marriage market, we have not tested it. This
  system is intended to be used with the the two-queue
  system.}




%%% TeX-master: "top"
