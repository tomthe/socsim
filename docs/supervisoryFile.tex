
\section{Supervisory and rate files}
\label{sec:supFile}

The name of the \voc{supervisory file}, generally with  \user{.sup}
suffix, is passed to Socsim on the command line.  Socsim expects such
a filename followed by a random number seed and will not run without
these two command line arguments.

What Socsim expects to find in the supervisory file is a set of
parameters and possibly rate specifications that allow it to run the
simulation.  The supervisory file has to either contain all of the
information Socsim needs (aside from the random number seed) or else
it must have \voc{include} directives that tell Socsim where else to
look. The \user{.sup} file must consist entirely of valid directives
or comment lines. Comments are lines that begin with '*'. Comments are
ignored by Socsim.

The \voc{.sup} file includes both global (affecting the entire
simulation) and segment specific simulation parameters.  The
\keyw{run} directive indicates the end of a set of segment specific
parameters.  When Socsim encounters a \keyw{run} directive, it stops
reading the \user{.sup} file and executes the simulation segment. When
the segment's execution is complete, Socsim returns to reading the
\user{.sup} file where it left off.  This has two important
implications:

\begin{enumerate}
\item If the directives for a  segment are not followed by a
  \keyw{run} directive, the segment will not be executed. Socsim will
  simply read the instructions for the next segment and execute those
  -- assuming that they end with a \keyw{run} directive.
\item Errors in a \user{.sup} file will not be caught until they are
  encountered. If there is an error in the specification of the
  $92^{nd}$ segment, Socsim will execute the first 91 segments before
  it exiting abnormally.
\end{enumerate}

\subsection{Minimal .sup file}
\label{sec:minSup}

At a minimum the supervisory file must include only a few directives.
Figure~\ref{fig:supSample} shows a minimal but sufficient \voc{.sup}
file. 
\begin{figure}[h]
  \centering
\vspace{.25cm}
\rule{.5\textwidth}{.1mm}
\begin{verbatim}
************************************************************************
** This is the simplest possible socsim .sup file. It will run a one
** segment simulation with starting population in test.opop and
** test.omar and ending population in test.out{.opop,.omar} The
** duration of the one and only "segment" is 1200 months; the minimum
** birth interval is 24 months; heterogenous fertility is turned off;
** rate files are in ratefile.Lese.
**
** type [path to socsim] run.sup 12345
** to run socsim
************************************************************************
segments 1
input_file test
output_file test.out
duration 1200

include ratefile.Lese
run

\end{verbatim}
  \caption{Minimal supervisory file}
  \label{fig:supSample}
\rule{.5\textwidth}{.1mm}
\end{figure}
\clearpage
\subsection{Global Directives}
\label{sec:globalDirectives}

Global directives can affect the entire simulation and
\emph{generally} do not change with each new segment. With the
exception of \keyw{segments}, \keyw{input\_file} and
\keyw{output\_file}, however, it is possible to change global
directives in a simulation segment. What ``global'' means is that
socsim does \textbf{not} reinitialize these directives to their
default settings at the beginning of each simulation segment--as it
does for the ``segment specific'' directives described in
Section~\ref{sec:segmentSpecific}.

This means that \textbf{you can change most global directives within
  each simulation segment}.  If Socsim finds a global directive such
as \keyw{bint} or \keyw{sex\_ratio} after the first \keyw{run}
directive, it \textbf{will} change its behavior accordingly and the
new behavior will persist until either the end of the simulation or
until the directive is encountered again.

\textbf{Be aware} that if you decide to change a global directive for
a particular segent, \textbf{the new value becomes the default} for
subsequent simulation segments.  In the case of segment specific
directives, the default values are reset at the start of a new
segment.

\begin{enumerate}
\item \dir{segments}{int}{0}{
\examp{segments 13}
The number of simulation
  \voc{segments} in the simulation run. A segment is a period of time
  under which a given set of vital rates are effective.  A simulation
  segment is executed when \keyw{run} directive is encountered. Since
  the rates are reinitialized at the begininng of each segment, it is
  necessary to (re)specify all rates in every segment.  The
  \keyw{include} directive makes this convenient.}

% \item \dir{firstyear}{n}: n is the year in which the simulation
%   begins. If this is not given then Socsim will find the most recent
%   event date in the input files and use the following month as the
%   start of the simulation.


\item \dir{input\_file}{word}{none} {
\examp{input\_file initpop}Indicates where Socsim should
    look for the input files. The arguement of this directive is a
    stem from which complete filename paths are constructed.
    \user{input-file-stem}.opop will be the initial population file;
    \user{input-file-stem}.omar will be the initial marriage file.
    The initial marriage file need not exist if the initial population
    has no kinship structure. }

\item \dir{output\_file}{word}{none} {
\examp{output\_file resultspop}Indicates
  where Socsim should write output files.  Just as with
  \keyw{input\_file} directive, Socsim expects the argument to be a
  file stem, from which Socsim can construct a complete path by simply
  appending a suffix.  By default, the population and marriages files
  are only written 
  at the end of the final segment and are called
  \user{output-file-stem}.opop and
  \user{output-file-stem}.omar. However, The \keyw{write\_output}
  directive below causes these files to be written at the end of a segment and
  modifies naming conventions to avoid overwriting.}

\item \dir{proportion\_male}{real}{0.5112}{
\examp{proportion\_male 0.5112}Indicates the proportion of
  births which are male. In order to preserve an early
    mistake, the directive \keyw{sex\_ratio} is a synonym for
    \keyw{proportion\_male}.}

\item \dir{hetfert}{1/0}{1}{
\examp{hetfert 1}Enables the heterogeneous fertility
  feature (0 to disables it). If enabled, each female will have a beta
  distributed random variable by which her fertility is multiplied
  before random waiting times are generated.  The result is a wider
  variation in sibling set size than would otherwise result. Note that
  the degree to which this fertility ``multiplier'' is inherited by
  daughters can be set by the user see \voc{alpha} and  \voc{beta}.}


\item \dir{alpha,beta}{real}{1.0,0}{
\examp{alpha 1.0}
\examp{beta 0.0}Determine the degree to which the fertility multiplier
  is inherited from each woman's mother. See Section~\ref{secfmult}
  for a complete description.}
% See
%   Section~\ref{sec:rateMult} for details on how the fertility and
%   other rate multipliers work.

\item \dir{random\_father}{0/1}{0}{
\examp{random\_father 0}
Indicates whether or not births to
  unmarried women should have a father randomly assigned.  The default
  value 0 means ``no''. In this case the children of
  nonmarried/non-cohabiting mothers will have 0 as their father's
  id.}


\item \dir{random\_father\_min\_age}{int}{15} {
\examp{random\_father\_min\_age  15}
Specifies the minimum
    age in years that males must be in order to be picked as fathers
    of children born to unmarried/non-cohabiging women.  Constraints
    on marriageability e.g. incest constraints and endogamy
    constraints are also enforced with respect to random fathers, but
    marital status is ignored.  The default value is 15. Obviously
    this is meaningless unless the \keyw{random\_father} is set to 1}


\item \dir{bint}{int}{9}{
\examp{bint 9}
The minimum birth interval. As a concession
  to reality, Socsim can impose a minimum number of months between
  births. For humans 9 months is a good number to use. Socsim adjusts
  the specified fertility rates upward to compensate for the birth
  interval.}


\item \dir{endogamy}{(-1..1)}{0}{
\examp{endogamy 0}
Determines how suitors from other
  \voc{groups} are treated in determining suitability for
  marriage. When ego is having a marriage event executed, s/he
  inspects everyone on the marriage queue of the opposite sex. 
    A value of  \keyw{endogamy}  between 0 and 1 is taken as the
  probability that a potential spouse who is a member of a \emph{different}
  group from that of ego will be rejected.  A value of 1 therefore
  implies \emph{endogamy} while a value of 0 implies that group
  membership will not matter with respect to marriage.
  A value of \keyw{endogamy} between -1 and 0 is taken as the negative
  of the probability that a potential spouse from the \textbf{same
    group} will be rejected.  A value of -1 therefore enforces
  complete \emph{exogamy} -- all suitors of ego's own group will be
  rejected with probability 1.

  The default value is 0 implies ``radnomogamy'' meaning that group
  membership is not considered when evaluating potential spouses.}

  % \begin{eqnarray}
  % \mbox{marriage\_agedif\_min} <= a <= \mbox{marriage\_agedif\_max}\\
  % \mbox{where}  a = \mbox{groom's age} - \mbox{bride's age}
  % \end{eqnarray}
 
% marriage_peak_age = 36;
%    marriage_slope_ratio =2;
%    marriage_agedif_max = 120;
%    marriage_agedif_min = -120;

\item \dir{marriage\_queues}{1 or 2}{2}{
\examp{marriage\_queues 2}
Determines which of the two
    possible marriage market schemes will be used. A ``1'' indicates
    that the \voc{one-queue} system will be employed. Under this
    system \voc{all marriage-eligible males} are evaluated for each
    female with a scheduled marriage event. A ``2'' indicates that the
    \voc{two-queue} marriage market system will be used. In that case,
    both males and females have stochastically scheduled marriage
    (search) events and both sexes wait in their respecitve marriage
    queues if no suitable partner is immediately available. See
    Section~\ref{sec:marriageQueue} for a longer explanation}


\item
  \dir{marriage\_eval}{preference/distribution}{preference}{
\examp{marriage\_eval preference}
Determines
    the method of evaluating potential marriages. ``preference''
    indicates that the legacy age difference preference schedule be
    used. That scheme favors marriages based on their closeness to an
    ideal age difference. (See \keyw{marriage\_peak\_age} and
    \keyw{marriage\_slope\_raio}). ``distribution''  indicates that socsim
    should attempt to match a target distribution of the age difference
    between spouses at marriage.  See \keyw{agedif\_marriage\_mean}
    and \keyw{agedif\_marriage\_sd}. Also see
    Section~\ref{sec:score3} for a lengthier explanation of all this.}

\item \dir{agedif\_marriage\_mean}{\emph{group} real}{\emph{all
      groups} 2.0}{ \examp{agedif\_marriage\_mean 1 2} Determines the
    mean (in \textbf{years} ) of the target distribution of spousal
    age differences for \textbf{\emph{women} of the specified group.}
    This directive consists of the word
    \keyw{agedif\_marriage\_age\_mean} followed by an integer
    indicating the group to which the directive applies and a real
    indicating the target mean spousal age difference in years.
    Currently Socsim uses a normal distribution as the target
    distribution.  We await theoretically robust arguments in favor
    other parametric distributions. This is \textbf{only valid if
      \keyw{marriage\_eval} is set to ``distribution''} See
    Section~\ref{sec:score3} for more details.}




\item \dir{agedif\_marriage\_sd}{int $real >0$}{all groups
    3}{
\examp{agedif\_marriage\_sd 1 3}
Determines the standard deviation of the target spousal age
    distribution. Just as with \keyw{agedif\_marriage\_mean}, this
    directive is specified separately for each group in the
    simulation.  This is \textbf{only valid if \keyw{marriage\_eval}
      is set to ``distribution''.}  See Section~\ref{sec:score3} for
    more details.}

\item \dir{marriage\_peak\_age}{int}{36}{
\examp{marriage\_peak\_age 36} Determines (together with
    \keyw{marriage\_slope\_ratio}) the preference for spousal age
    \textbf{difference} (in months) among spouses.  This is \textbf{only valid
      if \keyw{marriage\_eval} is set to ``preference''}.  See
    Section~\ref{sec:score3} for more details.}



\item \dir{marriage\_slope\_ratio}{real}{2}{
\examp{marriage\_slope\_ratio 2.0} Works with
    \keyw{marriage\_peak\_age} to determine a marriage preference
    ``score'' for potential spouses.  This is \textbf{only valid if
      \keyw{marriage\_eval} is set to ``preference''}.  See
    Section~\ref{sec:score3} for more details.}




  \item \dir{marriage\_agedif\_min}{int}{-120}{
      \examp{marriage\_agedif\_min -120}
      Determines the
      \emph{lower} end of the permissible age difference between
      spouses (groom age - bride age), in months.  This is
      \textbf{only valid if \keyw{marriage\_eval} is set to ``preference''}
       See Section~\ref{sec:score3} for
      more details.}

\item \dir{marriage\_agedif\_max}{int}{120}{
\examp{marriage\_agedif\_max 120}
Determines the
    \emph{upper} end of the permissible age difference between spouses
    (groom age - bride age), in months. This is
      \textbf{only valid if \keyw{marriage\_eval} is set to ``preference''}
       See Section~\ref{sec:score3} for
      more details.}


\item
  \dir{child\_inherits\_group}{rule}{from\_mother}{
\examp{child\_inherits\_group  from\_mother}
Determines how group membership
  is assigned at birth. Socsim understands the following rules:
  \begin{itemize}
  \item from\_mother
  \item from\_father
  \item from\_same\_sex\_parent
  \item from\_opposite\_sex\_parent  
  \item $n$ where $n$ is a group to which all new borns are assigned.  
  \end{itemize}

By default, Socsim assigns newborns to mother's group. That is, if no
option is specified, Socsim will use ``from\_mother''.}

\subsection{Directives used in extended versions}

A few directives are included for convenience when extending SOCSIM.  in the plain version of SOCSIM, none of these directives should be set.

\item \dir{parameter0..parameter5}{r}{none}{ \keyw{parameter0} through
  \user{parameter5} should be set only if the enhanced version of
  Socsim that you are using defines what they do.}

\item \dir{read\_xtra}{1/0}{0}{Causes Socsim read a file called
  \keyw{input-file-stem}.opox or crash if the file does not exist. The
  .opox file should contains a set of
  \keyw{extra variable} values for each person in the initial
  population.  This will generally only be used in versions of Socsim
  that have been modified.  See Section~\ref{sec:modification} for
  description of the programming hooks available for ``easy''
  modification.  The plain vanilla version of Socsim does not use
  extra variables so reading them will not generally cause anything
  useful to happen.}


\item \dir{size\_of\_extra}{int}{0}{Deterines is the number of variables that
  should be read for each person from the .opox file. This will
  generally be made to default when one modifies Socsim. In its
  unmodified state, Socsim does not use extra variables, so this directive
is very infrequently what you are looking for.}


\end{enumerate}






\subsection{Segment specific directives}
\label{sec:segmentSpecific}

These directives make sense if specified for each simulation segment.

\begin{enumerate}

\item \dir{duration}{int}{no default}{Determines the duration in
    months of the current simulaion segment.}





% \item \dir{duration\_specific}{demographic-rate-spec}{none}{This tells socsim to treat the specified
%     rate block as duration rather than age specific. Note that this
%     directive does not take the place of the keywords that introduce
%     the rate block.  To define a duration specific group transition,
%     something like the following is required:

% \begin{verbatim}
% duration_specific transit 4 M single 18
% transit 4 M single 18
% 5    0       0.0
% 5    10       0.5
% 0    1200    0.0 
% \end{verbatim}

% For details see  Section~\ref{sec:duratspectrans}.
% }
% \item \dir{hettrans}{1/0}: enables heterogeneous transition
%   probability. It only makes sense to set this to 1 if you are working
%   with an enhanced version of Socsim that defines how this should
%   work.






\item \dir{include}{filename}{none}{Socsim will read and parse
    \user{filename} as if it were part of the current file. It is
    strongly recommended that \keyw{include} be used to keep rate
    specifications separate from the rest of the simulation
    parameters.  See Section~\ref{sec:rateFiles} has details on how
    vital demographic rates are specified.}


\item \dir{execute}{Unix shell command}{none}{The specified command is
    passed to the Unix shell and its output is routed to the
    screen. After the command exits without error, Socsim returns to
    processing the current file.

  It is possible with this directive to call an external program that
  might perhaps generate rate sets on the fly
  perhaps in response to the previous simulation segment. For example:
}
\begin{verbatim}
  execute generate_rates 1 5 0 >mortality.seg4
  include mortality.seg4
\end{verbatim}

\end{enumerate}


%%% Local Variables: 
%%% mode: latex
%%% TeX-master: "top"
%%% End: 

% LocalWords:  Socsim
